\chapter{Introduction}
\label{chapter:intro}

\section{Brief history of vmchecker}
\label{sec:vmc-history}

\subsection{Motivation}
\label{sub-sec:vmc-history-motiv}

For students in Computer Science classes, programming assignments are the main
way in which they gain practical experience regarding certain concepts and 
technologies. These assignments are also an important opportunity to receive 
feedback and might also count for the student's final grade.

Student assignments are generally easy to assess. The process consists of 
downloading the student's submission, running a number of tests to check that 
it compiles and gives the right output and finally reviewing the source code
to make sure that no obvious mistakes were made.

While the whole process is not a difficult one, it quickly becomes a tedious task
when you need to assess more than 20 student assignments. Also, when a large number
of assignments need to be tested, they are generally divided between teaching 
assistants who do perform the tasks individually. Considering each of them might
test the assignments in a different environment, grading fairness and consistency
is eventually lost.

Another issue with the classical aproach to grading programming assignment is that
while providing the student with feedback upon the work he has done is one of the
main objectives, that feedback often arrives too late. Homework is usually graded
days or weeks after completion, so what would have been valuable advice during the
development phase, is now too little too late.

Apart from the code-review stage of assesing a student submission, the process is 
fairly easy to automate. Also, while an automated system may not provide feedback 
on the quality of the submitted code, it could provide the student with immediate
information regarding the behaviour of his homework given certain input data.

In order to automate the testing process as much as possible, the \project project 
was initiated by the Computer Science department at the Politehnica 
University of Bucharest.

This project aims to improve the performance of the vmchecker automated grading
system by adding support for multiple virtualization types, simplifying its
design and by insuring its scalability.

\subsection{Development and support}
\label{sub-sec:vmc-history-dev}

\todo{Info}

\subsection{Features}
\label{sub-sec:vmc-history-features}

\todo{Features}

\section{The architecture of vmchecker}
\label{sec:vmc-architecture}

The vmchecker automated grading system consists of two separate subsystems:
\begin{enumerate}
\item \nameref{sub-sec:storer}
\item \nameref{sub-sec:tester}
\end{enumerate}

\begin{center}
\fig[scale=.6]{src/img/simple-structure}{img:simple-structure}{Simple structure}
\end{center}

The storer subsystem runs an webserver that allow the user to upload an archive
containing the user's homework assignment. This archive is stored in a repository,
along with the results of the homework's evaluation.

The tester runs a queue manager, which awaits tasks from the storer and upon
receiving them proceeds to powering on a virtual machine, compiling the submitted
source code, running certain tests and uploading those tests to the storer.



\subsection{The storer}
\label{sub-sec:storer}

The storer is basically a Linux running machine that runs an Apache webserver
in order to present the student with a web-interface so he can upload his
assignment. As well as running the said webserver, the storer also hosts a 
repository of containing all of the student submissions, organized by course,
assignment, student name and submission date. This is done in order to prevent
the accidental overwriting of a submission.

\begin{center}
\fig[scale=.4]{src/img/storer-structure}{img:storer-structure}{Storer structure}
\end{center}

sdfsdfsdf

\begin{center}
\fig[scale=.4]{src/img/repo-structure}{img:repo-structure}{Repo structure}
\end{center}

sdsadas











\subsection{The tester}
\label{sub-sec:tester}

\begin{center}
\fig[scale=.6]{src/img/tester-structure}{img:tester-structure}{Tester structure}
\end{center}

\todo{abcd}

\section{Analysis of the shortcomings of the former architecture}
\label{sec:vmc-analysis}

\todo{abcd}

\section{Improving the existing model}
\label{sec:vmc-improving}

\todo{abcd}

