\chapter{Load Balancing and Scalability}
\label{chapter:virt-load}

One of the major drawbacks of the current vmchecker is that at times an entire
tester may be sitting idle, while another has over a dozen submissions to
evaluate. The reason for this is that each assignment's tester is statically
set in the course's \textit{config} file. 

The first problem to this approach is that testing a certain submission might
require certain packages to be installed on the host operating system, 
that certain virtual machines might not exist, or more generally that
evaluating a submission might fail, even though the submission should
succeed.

Another problem is the way you assign submissions to each tester queue. 
While the web interface is able to check how many submissions are in a queue,
it doesn't have any way of estimating the time an evaluation might take.

In order to tackle these issues, when notifing the storer that it has finished
evaluating an assignment, the tester will also provide the running time. 
Failing to contact the tester, to start the virtual machine or no receival of
the notification will be counted as taking a long time. This means that submissions
for that assignment will not be directed to that virtual machine.

Determining the compatible virtual machines will be done using the course's official
implementation. The submission will be added to every available queue, excepting
those that feature an incompatible operating system or virtualization environment.
For example, the submission to a Windows assignment will never be added to the
queue of a Linux virtual machine, also a kernel module will never be tested 
inside an LXC container.

\section{Estimating the Evaluation Time}
\label{sec:vmc-estimating}

Evaluating the time that evaluating a submission might take is a generally
straight-forward process. 

\section{Choosing the Best Machine}
\label{sec:vmc-best}





