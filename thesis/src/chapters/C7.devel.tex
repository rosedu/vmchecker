\chapter{Further development}
\label{chapter:virt-devel}

Although \project is a perfectly usable system, there is always room for improvement.
We have identified some of the areas where \project lacks support and intend to
develop them in the future.

\begin{itemize}
\item {\bf Web interface for grading the submissions}. Currently this is done by
remotely connecting to the storer, reviewing the source code and the test results
and writing the score in a text file. A web interface for grading and administration
would improve both the usability and security of the system.
\item {\bf Storer access for students}. Accidents where a student uploads a
homework in the wrong section happen with increased regularity. This is usually
fixed by a course administrator. A student's ability to manage his own submissions
would be a great improvement.
\item {\bf CLI utility for submitting assignments}. A command line tool for 
submitting the homework might prove very useful to many students, especially
those who prefer to work on systems without a graphical user interface.
\item {\bf Cache for LDAP authentication}. Presently students use their
LDAP user account and password to authenticate in the web interface. However,
if the LDAP server ever fails that means that vmchecker, although running,
cannot be accessed by students.
\item {\bf MPI and OpenMP support}. In order to integrate these testing platforms
into \project, one only needs to extend the generic executor and implement
the methods needed to interact with the system used for testing the MPI or OpenMP
assignment.
\item {\bf Debian installation package}. Installing \project on a new system
is a difficult task, especially for those who have never used it before. 
A package that would take care of the installation process would be a great 
step forward regarding the popularity of \project as an automatic grading system.
\item {\bf Libvirt-sandbox integration}. Using this library would provide
better tools for interaction with the virtual machines, especially for KVM,
where running a command on the guest operating system suffers from performance
issues due to the use of SSH.
\end{itemize}
